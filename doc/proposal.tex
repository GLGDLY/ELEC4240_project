% CVPR 2022 Paper Template
% based on the CVPR template provided by Ming-Ming Cheng (https://github.com/MCG-NKU/CVPR_Template)
% modified and extended by Stefan Roth (stefan.roth@NOSPAMtu-darmstadt.de)

\documentclass[10pt,twocolumn,letterpaper]{article}

%%%%%%%%% PAPER TYPE  - PLEASE UPDATE FOR FINAL VERSION
% \usepackage[review]{cvpr}      % To produce the REVIEW version
\usepackage{cvpr}              % To produce the CAMERA-READY version
%\usepackage[pagenumbers]{cvpr} % To force page numbers, e.g. for an arXiv version

% Include other packages here, before hyperref.
\usepackage{graphicx}
\usepackage{amsmath}
\usepackage{amssymb}
\usepackage{booktabs}


% It is strongly recommended to use hyperref, especially for the review version.
% hyperref with option pagebackref eases the reviewers' job.
% Please disable hyperref *only* if you encounter grave issues, e.g. with the
% file validation for the camera-ready version.
%
% If you comment hyperref and then uncomment it, you should delete
% ReviewTempalte.aux before re-running LaTeX.
% (Or just hit 'q' on the first LaTeX run, let it finish, and you
%  should be clear).
\usepackage[pagebackref,breaklinks,colorlinks]{hyperref}


% Support for easy cross-referencing
\usepackage[capitalize]{cleveref}
\crefname{section}{Sec.}{Secs.}
\Crefname{section}{Section}{Sections}
\Crefname{table}{Table}{Tables}
\crefname{table}{Tab.}{Tabs.}


\begin{document}

%%%%%%%%% TITLE 
\title{Project proposal for Text Eraser on Image}

\author{
Members: Leung Tsz Kit Gary (tkleungal, 20863153);  % Fill in name here
Ip Marisa (mip, 21054353)
}
\maketitle

%%%%%%%%% CONTENT 
\section{Background}

In this project, we aim to develop a model solution to perform the task of text removal on images. 
The idea was initiated from the problem that most of the text eraser solutions online currently are performed with English word processing, 
with bad performance on processing of Chinese characters mixed with English characters. 
However, in the Hong Kong environment, such case is a common scenario. Therefore, we would like to develop a solution that can handle such use case.

\section{Proposed Methodology}

\subsection{Implementation}

It is proposed that the machine learning solution will be split into 2 parts:
\begin{enumerate}
    \item Image text segmentation
    \item Inpainting
\end{enumerate}

For the flow of the solution, a network of image text segmentation will first be computed with the input image.
The output of the segmentation will then be a mask of the text region.
This mask can then be used for the inpainting model to remove the text region from the image.
Resulting in the final output image with the text removed.

\subsubsection{Image text segmentation}

\subsubsection{Inpainting}

\subsection{Evaluation}

\end{document}
